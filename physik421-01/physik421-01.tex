\documentclass[11pt, ngerman, fleqn, DIV=15, headinclude]{scrartcl}

\usepackage[bibatend, color]{../header}

\hypersetup{
    pdftitle=
}

\renewcommand{\thesubsection}{\thesection.\alph{subsection}}

%\subject{}
\title{Quantenmechanik, Blatt 1}
%\subtitle{}
\author{
    Frederike Schrödel \and Jan Weber \and Simon Schlepphorst
}

\usepackage{mathtools}
\newcommand{\abs}[1]{\left\lvert#1\right\rvert}

\begin{document}

\maketitle

\setcounter{section}{1}

\section{Erinnerung Wahrscheinlichkeitsrechnung}

\subsection{}

Gegeben ist die die Gausverteilung mit:

\begin{align*}
  f\del{x}=\frac1{a_0\sqrt{2\pi}}\eup^{-\frac{x^2}{2a_0^2}}
\end{align*}

Gesucht sind Mittelwert und Varianz.\\
Der Mittelwert ist für kontinuierliche Funktionen gegeben durch:

\begin{align*}
  \bracket{x}	&=\int_{-\infty}^{\infty}xf\del{x}\dif x\\
		&=\int_{-\infty}^{\infty}\frac{x}{a_0\sqrt{2\pi}}\eup^{-\frac{x^2}{2a_0^2}}\dif x\\
		&=\left[-\frac{a_0}{\sqrt{2\pi}}\eup^{-\frac{x^2}{2a_0^2}}\right]_{-\infty}^{\infty}\\
		&=0
\end{align*}

Die Varianz einer kontinuierlichen Funktion ist gegeben durch

\begin{align*}
  \del{\Delta x}^2	&=\int_{-\infty}^{\infty}x^2f\del{x}\dif x-\del{\int_{-\infty}^{\infty}xf\del{x}\dif x}^2\\
  					&=\int_{-\infty}^{\infty}\frac{x^2}{a_0\sqrt{2\pi}}\eup^{-\frac{x^2}{2a_0^2}}\dif x\\
  					&=\int_{-\infty}^{\infty}x\frac{x}{a_0\sqrt{2\pi}}\eup^{-\frac{x^2}{2a_0^2}}\dif x\\
  					&=\left[-\frac{xa_0}{\sqrt{2\pi}}\eup^{-\frac{x^2}{2a_0^2}}\right]_{-\infty}^{\infty} + \int_{-\infty}^{\infty}\frac{a_0}{\sqrt{2\pi}}\eup^{-\frac{x^2}{2a_0^2}}\dif x\\
  \intertext{%
  	Substituiere $y=\frac{x}{a_0}\Leftrightarrow\dif x=a_0\dif y$
  }
  					&=\int_{-\infty}^{\infty}\frac{a_0^2}{\sqrt{2\pi}}\eup^{-\frac{y^2}{2}}\dif y\\
 \intertext{Mit $\frac1{2a_0^2} > 0$ gilt:}
 &=\frac{a_0}{\sqrt{2\pi}} \sqrt{\frac{\pi}{\frac12 a_0^2}}\\
 &=a_0^2
\end{align*}


\subsection{}

Gegeben ist die Poissonverteilung mit:

\begin{align}
	f_\lambda\del k = \lambda^k \frac{\eup^{-\lambda}}{k!} \quad\text{ mit } k =
	0,1,2,\ldots,\infty
\end{align}

Der Mittelwert ist gegeben durch:
\begin{align*}
	\bracket{x} &= \sum_{i=1}^\infty f_\lambda\del k i\\
	&= \sum_{i=1}^\infty \frac{\lambda^i \eup^{-\lambda}}{i!}i\\
	&= \sum_{i=0}^\infty \frac{\lambda^{i+1}
	\eup^{-\lambda}}{\del{i+1-1}!}\\
	&= \lambda \eup^{-\lambda} \sum_{i=0}^\infty \frac{\lambda^i}{i!}\\
	&= \lambda
\end{align*}

Die Varianz ist gegeben durch:
\begin{align*}
	\del{\Delta x} &= \sum_{i=1}^\infty \frac{\lambda^i
		\eup^{-\lambda}}{i!} i^2 - \del{\sum_{i=1}^\infty
		\frac{\lambda^i \eup^{-\lambda}}{i!} i}^2\\
		&= \sum_{i=1}^\infty \frac{\lambda^i
			\eup^{-\lambda}}{\del{i-1}!} i - \lambda^2\\
			&= \lambda \eup^{-\lambda} \sum_{i=0}^\infty
			\frac{\lambda^i}{i!} \del{i+1} - \lambda^2\\
		&= \lambda + \lambda \eup^{-\lambda}\del{ \frac{\lambda^0 0}0 +
		\sum_{i=1}^\infty \frac{\lambda^{i-1}}{\del{i-1}!} }
		-\lambda^2\\
	\intertext{Mit $n := i-1$ ergibt sich:}
	&= \lambda + \lambda^2 \eup^{-\lambda} \sum_{n=0}^\infty
	\frac{\lambda^n}{n!} - \lambda^2\\
	&= \lambda
\end{align*}

\subsection{}


\section{Fouriertransformation}


\section{Unschärferelation}

\subsection*{4.A}

Gegeben sind die folgenden Relationen:

\begin{align*}
	\braket{p_x} &= \int_{-\infty}^{\infty} \dif^3\vec{p}\; p_x \abs{g\del{\vec{p}}}\\
	\braket{p_x^2} &= \int_{-\infty}^{\infty} \dif^3\vec{p}\; p_x^2 \abs{g\del{\vec{p}}}
\end{align*}

Analog für $\bracket{x}$ und $\bracket{x^2}$ für $f\del{\vec{r}}$.\\

Gezeigt werden soll die Ort-Zeitunschärfe in der Form

\begin{align*}
	\Delta x\Delta p_x \geq \frac{\hbar}{2}
\end{align*}

Dabei ist der folgende Ansatz geben:

\begin{align*}
	I\del{\lambda} = \int \dif k\; \abs{kf\del{k}+\lambda\od{f}{k}}^2
\end{align*}

Sei $\psi$ eine Wellenfunktion im Ortsraum und $\fourier\psi$ die Fouriertransformierte von $\psi$ im Impulsraum. Für ein eindimensionales Wellenpaket folgt:
\begin{align*}
	I\del{\lambda} 	&= \int\dif x\; \abs{x\psi + \lambda\pd{\psi}{x}}^2  \\
					&= \int\dif x\; x^2\psi^*\psi + \lambda \int\dif x\; x\del{\psi^*\pd{\psi}{x} + \psi\pd{\psi^*}{x}} + \lambda^2 \int\dif x\; \pd{\psi^*}{x}\pd{\psi}{x}\\
	\intertext{%
		Partielle Integration des mittleren Integrales gibt:
	}
					&= \int\dif x\; \psi^*x^2\psi - \lambda \underbrace{\int\dif x\; \psi^*\psi}_{=1} - \lambda^2 \int\dif x\; \pd{\psi^*}{x}\pd{\psi}{x}\\
	\intertext{%
		Nach dem Theorem von Parselval-Plancherel gilt:
	}
					&= \int\dif x\; \psi^*x^2\psi - \lambda - \lambda^2 \int\dif x\; \fourier\del{\pd{\psi^*}{x}}\fourier\del{\pd{\psi}{x}}\\
	\intertext{%
		Nach Aufgabenteil 3B gilt
	}
					&= \int\dif x\; \psi^*x^2\psi - \lambda + \lambda^2 \int\dif x\; k^2\fourier\del{\psi^*}\fourier\del{\psi}\\
	\intertext{%
		Da die Varianz der Welllenfunktion sowohl im Orts-, wie auch im Impulsraum verschwindet folgt:
	}
					&= \del{\Delta x}^2 - \lambda + \lambda^2 \del{\Delta k}^2\\
\end{align*}

Mit $k=\frac{p_x}{\hbar}$ folgt:
\begin{align}
	I\del{\lambda} = \del{\Delta x}^2 - \lambda + \frac{\lambda^2}{\hbar^2} \del{\Delta p_x}^2 \overset{!}{\geq} 0
\end{align}

$I\del{\lambda}$ ist eine Parabel. Für den Scheitelpunkt ($\od{I\del{\lambda}}{\lambda} = 0$) gilt:

\begin{align*}
	\od{I\del{\lambda}}{\lambda} = -1 + \frac{2\lambda}{\hbar^2}\del{\Delta p_x}^2\overset{!}{=}0 \Leftrightarrow \lambda\left.\right|_{\od{I\del{\lambda}}{\lambda} = 0} \coloneqq \lambda_S = \frac{\hbar^2}{2\del{\Delta p_x}^2}
\end{align*}

Einsetzen in (1) gibt:
\begin{align*}
	I\del{\lambda_S} = \del{\Delta x}^2 - \frac{\hbar^2}{2\del{\Delta p_x}^2} + \frac{\hbar^2}{4\del{\Delta p_x}^2} \overset{!}{\geq} 0 \Leftrightarrow \del{\Delta x}^2\del{\Delta p_x}^2 \geq \frac{\hbar^2}4
\end{align*}

Und somit folgt als Endergebnis die Orts-Impulsunschärfe mit:
\begin{align*}
	\del{\Delta x}\del{\Delta p_x} \geq \frac{\hbar}2
\end{align*}



\subsection*{4.B}

Gegeben ist die Gaussfunktion mit

\begin{align*}
	b
\end{align*}



\end{document}

% vim: spell spelllang=de tw=79
