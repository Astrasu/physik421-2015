\documentclass[11pt, ngerman, fleqn, DIV=15, headinclude]{scrartcl}

\usepackage[bibatend, color]{../header}

\hypersetup{
    pdftitle=
}

\renewcommand{\thesubsection}{\thesection.\alph{subsection}}

\usepackage{listings}
\usepackage{beramono}
\lstset{
    basicstyle=\small\tt
}

%\subject{}
\title{Quantenmechanik, Blatt 5}
%\subtitle{}
\author{
    Frederike Schrödel \and Heike Herr \and Jan Weber \and Simon Schlepphorst
}


\begin{document}

\maketitle

\section{Potentialstufen}

\section{Gebundene Zustände eines $\delta$-Potentials}

Betrachtet wird ein Teilchen mit Masse $m$, dass sich in einem Potential
\begin{align*}
	V\del{x}=V_0\delta\del{x}&&V_0<0
\end{align*}
bewege.\\
Es sollen die gebundenen Zustände und Eigenenergie bestimmt werden.\\
Ansatz:
\begin{align*}
	\psi_-	&= A\eup^{qx}+A'\eup^{-qx}\\
	\psi_+	&= B'\eup^{qx}+B\eup^{-qx}
\end{align*}
Dabei ist
\begin{align*}
	q=\sqrt{\frac{-2mE}{\hbar^2}}&&E<0
\end{align*}
Da die Wellenfunktionen $\psi_-$ und $\psi_+$ normierbar sein sollen muss gelten:
\begin{align*}
	\psi_-	&= A\eup^{qx}\\
	\psi_+	&= B\eup^{-qx}
\end{align*}
Aufgrund der Stetigkeit muss
\begin{align*}
	&\psi_-\del{0}=\psi_+\del{0}\\
	&\Rightarrow A=B
\end{align*}
sein.\\
Um den Normierungsfaktor auszurechnen bilde ich das Skalarprodukt:
\begin{align*}
	\inner{\psi}{\psi}	&= \int_{-\infty}^{\infty}\dif x\;\psi^*\psi\\
						&= \abs{A}^2\del{\int_{-\infty}^0\dif x\;\eup^{2qx} + \int_0^{\infty}\dif x\;\eup^{-2qx}}\\
						&= \abs{A}^2\del{\sbr{\frac1{2q}\eup^{2qx}}_{-\infty}^0 - \sbr{\frac1{2q}\eup^{-2qx}}_0^{\infty}}\\
						&= \frac{\abs{A}^2}{q}
	\intertext{%
		Wegen $\inner{\psi}{\psi}\overset{!}{=}1$ folgt
	}
						\Rightarrow A&=\sqrt{q}
\end{align*}
Also:
\begin{align*}
	\psi_-	&= \sqrt{q}\eup^{qx}\\
	\psi_+	&= \sqrt{q}\eup^{-qx}
\end{align*}
Um die Energie zu bestimmen löse ich die Schrödingergleichung:
\begin{align*}
	&\frac{\hbar^2}{2m}\psi'' + E\psi = V_0\delta\del{x}\psi\\
	\intertext{%
		Integration von $\-\epsilon$ bis $\epsilon$ über $x$ und bilden des Limes gibt:
	}
	\Leftrightarrow&\lim_{\epsilon\to0}\del{\int_{-\epsilon}^{\epsilon}\dif x\;\frac{\hbar^2}{2m}\psi'' + \int_{-\epsilon}^{\epsilon}\dif x\;E\psi} = \lim_{\epsilon\to0}\int_{-\epsilon}^{\epsilon}\dif x\;V_0\delta\del{x}\psi\\
	\intertext{%
		Die Integration beim zweiten Integral gibt nach dem ausführen des Limes null für den Rest ergibt sich:
	}
	\Leftrightarrow&\frac{\hbar^2}{2m}\lim_{\epsilon\to0}\del{\psi'\del{\epsilon}-\psi'\del{-\epsilon}}=V_0\psi\del{0}\\
	\Leftrightarrow&\frac{\hbar^2}{2m}\lim_{\epsilon\to0}\del{-\sqrt{q}q\eup^{-q\epsilon}-\sqrt{q}q\eup^{-q\epsilon}}=V_0\sqrt{q}\\
	\Leftrightarrow&-\frac{\hbar^2q}{m}=V_0\\
	\intertext{%
		Einsetzen von $q$ gibt:
	}
	\Leftrightarrow&-\frac{2E\hbar^2}{m}=V_0^2\\
	\Leftrightarrow&E=-\frac{mV_0^2}{2\hbar^2}\\
\end{align*}

\section{Gebundene Zustände von zwei $\delta$-Potentialen}

\section{Adjungieren}

Gegeben sind mehrere Ausdrücke. Es sollen der Typ und das Adjungierte bestimmt werden.

\subsection{}
\begin{align*}
	\del{\hat{A}+\lambda\hat{B}^{\dagger}}\ket{\psi_1}
\end{align*}
ist ein Zustand, das Adjungierte ist:
\begin{align*}
	\bra{\psi_1}\del{\hat{A}^{\dagger}+\lambda^*\hat{B}}
\end{align*}

\subsection{}
\begin{align*}
	\hat{A}\ket{\psi_1}\bra{\psi_2}\lambda\hat{B}\hat{C}
\end{align*}
ist ein Operator, das Adjungierte ist:
\begin{align*}
	\lambda^*\hat{C}^{\dagger}\hat{B}^{\dagger}\ket{\psi_2}\bra{\psi_1}\hat{A}^{\dagger}
\end{align*}

\subsection{}
\begin{align*}
	\bra{\psi_1}\hat{A}\ket{\psi_2}\ket{\psi_1}
\end{align*}
ist ein Zustand, das Adjungierte ist:
\begin{align*}
	\bra{\psi_1}\bra{\psi_2}\hat{A}^{\dagger}\ket{\psi_1}
\end{align*}

\subsection{}
\begin{align*}
	\bra{\psi_2}\hat{A}\hat{B}^{\dagger}\hat{C}^{\dagger}\ket{\psi_1}
\end{align*}
ist ein Skalar, das Adjungierte ist:
\begin{align*}
	\bra{\psi_1}\hat{C}\hat{B}\hat{A}^{\dagger}\ket{\psi_2}
\end{align*}

\subsection{}
\begin{align*}
	\bra{\psi_1}\del{\hat{C}+\hat{D}^{\dagger}}\del{\hat{A}-\iup\hat{B}}\ket{\psi_2}
\end{align*}
ist ein Skalar, das Adjungierte ist:
\begin{align*}
	\bra{\psi_2}\del{\hat{A}^{\dagger}+\iup\hat{B}^{\dagger}}\del{\hat{C}^{\dagger}+\hat{D}}\ket{\psi_1}
\end{align*}

\end{document}

% vim: spell spelllang=de tw=79
