\documentclass[11pt, ngerman, fleqn, DIV=15, headinclude]{scrartcl}

\usepackage[bibatend, color]{../header}

\hypersetup{
    pdftitle=
}

\renewcommand{\thesubsection}{\thesection.\alph{subsection}}

\usepackage{units}
\usepackage{listings}
\usepackage{beramono}
\lstset{
    basicstyle=\small\tt
}

%\subject{}
\title{Quantenmechanik, Blatt 7}
%\subtitle{}
\author{
    Frederike Schrödel \and Heike Herr \and Jan Weber \and Simon Schlepphorst
}



\begin{document}

\maketitle
\begin{center}
	\begin{tabular}{l|c|c|c|c|c}
		Aufgabe &1&2&3&4&$\Sigma$\\
		\hline
		Punkte &\quad /12 & \quad /10 & \quad /4 & \quad /14 & \quad /40
	\end{tabular}\\
\end{center}




\section{Neutrino Oszillationen}

Betrachtet wird ein Zwei-Niveau-System für die Entstehung von zwei Neutrinos $\nu_e$ und $\nu_{\mu}$ als Mischzustände zweier Eingenfunktionen $\ket{\nu_1}$ und $\ket{\nu_2}$ des Hamilton-Operators mit Mischwinkel $\theta$ durch:
\begin{align*}
	\begin{pmatrix}
		\ket{\nu_e}\\
		\ket{\nu_{\mu}}
	\end{pmatrix}
	=
	\begin{pmatrix}
		\cos\del{\theta}&\sin\del{\theta}\\
		-\sin\del{\theta}&\cos\del{\theta}
	\end{pmatrix}
	\begin{pmatrix}
		\ket{\nu_1}\\
		\ket{\nu_2}
	\end{pmatrix}
\end{align*}

\subsection{}

Es soll der Hamilton-Operator in der Basis $\ket{\nu_1}$ und $\ket{\nu_1}$ dargestellt werden:
\begin{align*}
	H_0=\ket{\nu_1}\bra{\nu_1}E_1 + \ket{\nu_2}\bra{\nu_2}E_2
\end{align*}

\subsection{}

Betrachtet wird ein Neutrino, welches zum Zeitpunkt $t=0$ erzeugt wird.Es soll die Zeitliche Entwicklung $\ket{\psi\del{t}}$ in der Basis von $H_0$ angegeben werden.\\
Für $\ket{\nu_e}$ ergibt sich nach obiger Formel:
\begin{align*}
	\ket{\nu_e}=\cos\del{\theta}\ket{\nu_1} + \sin\del{\theta}\ket{\nu_2}
\end{align*}
Die Zeitentwicklung ist durch den Zeitentwicklungsoperator
\begin{align*}
	u_n\del{t}=\eup^{-\iup\frac{E_n}{\hbar}t}
\end{align*}
gegeben, so dass sich für $\ket{\psi\del{t}}$ ergibt:
\begin{align*}
	\ket{\psi\del{t}}&=u_n\del{t}\ket{\nu_e}\\
	&=\cos\del{\theta}u_n\del{t}\ket{\nu_1} + \sin\del{\theta}u_n\del{t}\ket{\nu_2}\\
	&=\cos\del{\theta}\eup^{-\iup\frac{E_1}{\hbar}t}\ket{\nu_1} + \sin\del{\theta}\eup^{-\iup\frac{E_2}{\hbar}t}\ket{\nu_2}
\end{align*}

\subsection{}

Nun wird nach der Wahrscheinlichkeit gefragt zum Zeitpunkt $t$ das Teilchen im Zustand $\ket{\nu_{\mu}}$ zu finden.\\
Dazu berechne ich das Betragsquadrat des Skalarproduktes der beiden Zustände wobei ich die Zeitentwicklung durch $u_n\del{t}$ ausdrücke:
\begin{align*}
	P_{\nu_{\mu}}&= \abs{\bra{\nu_e}u_n\del{t}\ket{\nu_{\mu}}}^2\\
	&= \abs{\del{\cos\del{\theta}\bra{\nu_1} + \sin\del{\theta}\bra{\nu_2}}u_n\del{t}\del{-\sin\del{\theta}\ket{\nu_1} + \cos\del{\theta}\ket{\nu_2}}}^2\\
	&= \abs{-\sin\del{\theta}\cos\del{\theta}u_1\del{t} + \sin\del{\theta}\cos\del{\theta}u_2\del{t}}^2\\
	&= \sin^2\del{\theta}\cos^2\del{\theta}\abs{u_2\del{t} - u_1\del{t}}^2\\
	&= \sin^2\del{\theta}\cos^2\del{\theta}\del{u^*_2\del{t} - u^*_1\del{t}}\del{u_2\del{t} - u_1\del{t}}\\
	&= \sin^2\del{\theta}\cos^2\del{\theta}\del{1 - u^*_2\del{t}u_1\del{t} - u^*_1\del{t}u_2\del{t} + 1}\\
	&= \sin^2\del{\theta}\cos^2\del{\theta}\del{2-2\cos\del{\frac{E_2 - E_1}{\hbar}t}}
\end{align*}
Außerdem soll die Zeit $T$ angegeben werden zu welcher $P_{\nu_{\mu}}$ maximal wird. Dazu muss gelten:
\begin{align*}
	&\cos\del{\frac{E_2 - E_1}{\hbar}t}\overset{!}{=}-1\\
	\Leftrightarrow&\frac{E_2 - E_1}{\hbar}T_{\text{max}}=\del{2n+1}\pi\\
	\Leftrightarrow&T_{\text{max}}=\frac{\del{2n+1}\pi\hbar}{E_2 - E_1}
\end{align*}

\subsection{}

Das Neutrino ist ein relativistisches Teilchen. Für die Energie in Abhängigkeit von Impuls $p$ und Masse $m$ gilt:
\begin{align*}
	E^2 = p^2c^2 + m^2c^4
\end{align*}
Es soll die in Aufgabenteil 1.c. berechnete Wahrscheinlichkeit $P_{\nu_{\mu}}$ in Abhängigkeit von Impuls und Masse ausgedrückt werden. Dazu berechne ich $E_2 - E_1$:
\begin{align*}
	E_2 - E_1 	&= \sqrt{p^2c^2 + m_2^2c^4} - \sqrt{p^2c^2 + m_1^2c^4}\\
				&= \del{\sqrt{p^2c^2 + m_2^2c^4} - \sqrt{p^2c^2 + m_1^2c^4}}\frac{\sqrt{p^2c^2 + m_2^2c^4} + \sqrt{p^2c^2 + m_1^2c^4}}{\sqrt{p^2c^2 + m_2^2c^4} + \sqrt{p^2c^2 + m_1^2c^4}}\\
				&= \frac{p^2c^2 + m_2^2c^4 - p^2c^2 - m_1^2c^4}{\sqrt{p^2c^2 + m_2^2c^4} + \sqrt{p^2c^2 + m_1^2c^4}}\\
				&= \frac{\del{m_2^2 - m_1^2}c^4}{\sqrt{p^2c^2 + m_2^2c^4} + \sqrt{p^2c^2 + m_1^2c^4}}\\
	\intertext{%
		Mit der Näherung $pc>>mc^2$ folgt:
	}
				&= \frac{\del{m_2^2 - m_1^2}c^4}{2pc}\\
\end{align*}
Also ergibt sich für $P_{\nu_{\mu}}$:
\begin{align*}
	P_{\nu_{\mu}} = \sin^2\del{\theta}\cos^2\del{\theta}\del{2-2\cos\del{\frac{\del{m_2^2 - m_1^2}c^4}{2p\hbar c}t}}
\end{align*}

\subsection{}

Nach einer Distanz $l$ wird der Zustand gemessen. Es soll die Wahrscheinlichkeit $P_{\nu_{\mu}}$ in Abhängigkeit von $l$ dargestellt werden. Dabei bewege sich das Teilchen mit der Geschwindigkeit $c\del{1+\epsilon}$.\\
Es gilt:
\begin{align*}
	v = \frac{s}{t}\Leftrightarrow t = \frac{s}{v} = \frac{l}{c\del{1+\epsilon}}
\end{align*}
Somit folgt:
\begin{align*}
	P_{\nu_{\mu}} = \sin^2\del{\theta}\cos^2\del{\theta}\del{2-2\cos\del{\frac{\del{m_2^2 - m_1^2}c^4}{2p\hbar c^2\del{1+\epsilon}}l}}
\end{align*}

\subsection{}

Es wird ein Mischungswinkel von $\theta = \frac{\pi}{4}$ angenommen. Es soll die Länge $l$ bestimmt werden bei der die Wahrscheinlichkeit $P_{\nu_{\mu}}$ maximal ist.\\
Nach Aufgabenteil 1.e. gilt:
\begin{align*}
	P_{\nu_{\mu}} = \del{\half-\half\cos\del{\frac{\del{m_2^2 - m_1^2}c^4}{2p\hbar c^2\del{1+\epsilon}}l}}
\end{align*}
$P_{\nu_{\mu}}$ ist maximal für:
\begin{align*}
	&\cos\del{\frac{\del{m_2^2 - m_1^2}c^4}{2p\hbar c^2\del{1+\epsilon}}l_{\text{max}}}\overset{!}{=}\del{2n+1}\pi\\
	\Leftrightarrow&l_{\text{max}}=\frac{2p\hbar c^2\del{1+\epsilon}}{\del{m_2^2 - m_1^2}c^4}\del{2n+1}\pi
	\intertext{%
		Mit $\epsilon\approx0$, $\del{m_2^2 - m_1^2}c^4 = 1\del{\text{eV}}^2$, $pc = 10^{10}\del{\text{eV}}^2$ und $\hbar c = 2\cdot10^{-7}$eV$\cdot$m folgt:
	}
	\Leftrightarrow&l_{\text{max}}=\frac{10^{10}\del{\text{eV}}^2\cdot10^{-7}\text{eV$\cdot$ m}}{1\del{\text{eV}}^2}\del{2n+1}\pi = \del{2n+1}\pi\cdot10^3\text{m}
\end{align*}

\section{Atomspringbrunnen}

\subsection{}

	Wir berechnen $|\psi(T)\rangle$ in drei Schritten:
	\begin{itemize}
	\item Wir haben gegeben: $|\psi(0)\rangle=|1\rangle$. Dies geht zuerst durch den Resonator:
		\begin{align*}
			|\psi(\varepsilon)\rangle&=\frac{1}{\sqrt{2}} \begin{pmatrix}1 &-i\cdot 1 \\ -i\cdot 1 & 1 \end{pmatrix} \begin{pmatrix} 1 \\ 0 \end{pmatrix} \\
					&= \frac{1}{\sqrt{2}} |1\rangle -i\frac{1}{\sqrt{2}}   |2\rangle \\
		\end{align*}	
	
	\item  Nun verwenden wir den Zeitentwicklungsoperator, um $|\psi(T-\varepsilon)\rangle$ zu berechnen:
		\begin{align*}
			|\psi(T-\varepsilon)\rangle&= \hat{U}(T-\varepsilon,\varepsilon)|\psi(\varepsilon)\rangle \\
				&= \frac{1}{\sqrt{2}}e^{-i\frac{E_1(T-2\varepsilon)}{\hbar}}|1\rangle-i\frac{1}{\sqrt{2}}e^{-i\frac{E_2(T-2\varepsilon)}{\hbar}}|2\rangle
		\end{align*}
	\item Im letzten Zeitabschnitt geht die Welle ein weiteres Mal durch den Resonator:
		\begin{align*}
			|\psi(T)\rangle=&\frac{1}{\sqrt{2}}(\frac{1}{\sqrt{2}}e^{-i\frac{E_1(T-2\varepsilon)}{\hbar}} - \frac{1}{\sqrt{2}} e^{-i\omega (T-\varepsilon)}e^{-i\frac{E_2(T-2\varepsilon)}{\hbar}})|1\rangle  \\
			&+ \frac{1}{\sqrt{2}} (-i\frac{1}{\sqrt{2}}e^{-i\frac{E_1(T-2\varepsilon)}{\hbar}}e^{i\omega(T-\varepsilon)}-i\frac{1}{\sqrt{2}}e^{-i\frac{E_2(T-2\varepsilon)}{\hbar}})|2\rangle \\
			=&\frac{1}{2}(e^{-i\frac{E_1(T-2\varepsilon)}{\hbar}} -  e^{-i\omega (T-\varepsilon)}e^{-i\frac{E_2(T-2\varepsilon)}{\hbar}})|1\rangle \\
			&-i\frac{1}{2}(e^{-i\frac{E_1(T-2\varepsilon)}{\hbar}}e^{i\omega(T-\varepsilon)}+e^{-i\frac{E_2(T-2\varepsilon)}{\hbar}})|2\rangle
		\end{align*}
	\item Nun betrachten wir den Grenzfall $\varepsilon \rightarrow 0$ mit $E_1=\frac{\hbar\omega_0}{2}$ und $E_2=-\frac{\hbar\omega_0}{2}$:
		\begin{align*}
			\lim_{\varepsilon\rightarrow 0}|\psi(T)\rangle=&
				\frac{1}{2}(e^{-i\frac{E_1T}{\hbar}} -  e^{-i\omega T}e^{-i\frac{E_2T}{\hbar}})|1\rangle 
			-i\frac{1}{2}(e^{-i\frac{E_1T}{\hbar}}e^{i\omega T}+e^{-i\frac{E_2T}{\hbar}})|2\rangle \\
			=& i e^{-i\frac{\omega T}{2}} \frac{1}{2i} (e^{-i\frac{\omega_0T}{2}}e^{i\frac{\omega T}{2}}-e^{i\frac{\omega_0T}{2}}e^{-i\frac{\omega T}{2}})|1\rangle 
			-ie^{i\frac{\omega T}{2}} \frac{1}{2}(e^{-i\frac{\omega_0T}{2}}e^{i\frac{\omega T}{2}}+e^{i\frac{\omega_0T}{2}}e^{-i\frac{\omega T}{2}})|2\rangle \\
			=& i e^{-i\frac{\omega T}{2}} \sin((\omega-\omega_0)\frac{T}{2})|1\rangle 
			-ie^{i\frac{\omega T}{2}} \cos((\omega-\omega_0)\frac{T}{2})|2\rangle
		\end{align*}

	\end{itemize}


\subsection{}
	Die Wahrscheinlichkeit $P_2(\omega)$ ergibt sich wie folgt (für den Grenzfall $\varepsilon\rightarrow0$):
	\begin{align*}
		P_2(\omega)=& |\langle\psi(T)|2\rangle|^2 \\
				=&\cos^2((\omega_0-\omega)\frac{T}{2})
	\end{align*}	

	Dadurch erhalten wir als Halbwertsbreite $\Delta\omega$ von $P_2(\omega)$ um $\omega_0$:
	\begin{eqnarray*}
		P_2(\Delta\omega)=\frac{1}{2} & \Longleftrightarrow & \cos^2(\Delta\omega)\frac{T}{2})=\frac{1}{2} \\
		&\Longleftrightarrow & \Delta\omega\frac{T}{2}=\frac{\pi}{2} \\
		&\Longleftrightarrow & \Delta\omega=\frac{\pi}{T}
	\end{eqnarray*}
	
	Für einen 1Meter hohen Brunnen erhalten wir:
	\[ T=2\cdot \sqrt{\frac{2\cdot 1m}{g}} =2\cdot\sqrt{\frac{2m}{9,81\frac{m}{s^2}}}=0,90s \]
	Daraus ergibt sich:
	\[ \Delta\omega=\frac{\pi}{0,90s}=3,48s^{-1} \]


\section{GPS}


\section{Stimulierte Emission und Absorption}

\subsection{}
	Wir machen den Ansatz für $|\psi(t)\rangle=\begin{pmatrix}a(t)\\b(t)\end{pmatrix}$:
	\begin{align*}
		a(t)&=e^{-i\frac{(E_0-A)t}{\hbar}}\alpha(t) \\
		b(t)&=e^{-i\frac{(E_0+A)t}{\hbar}}\beta(t) \\
	\end{align*}
 	Wenn wir dies in die Schrödingergleichung $i\hbar\frac{d}{dt}|\psi(t)\rangle=\hat{H}|\psi(t)\rangle$ einsetzen erhalten wir die Gleichungen:
	\begin{align*}
	I.: &&	i\hbar(-i\frac{E_0-A}{\hbar}a(t)+e^{-i\frac{(E_0-A)t}{\hbar}}\dot{\alpha}(t))&=(E_0-A)a(t)-\hbar\omega_1 \cos(\omega t) e^{-i\frac{(E_0+A)t}{\hbar}}\beta(t) \\
	II.:&&	i\hbar(-i\frac{E_0+A}{\hbar}b(t)+e^{-i\frac{(E_0+A)t}{\hbar}})\dot{\beta}(t)&=-\hbar\omega_1\cos(\omega t) e^{-i\frac{(E_0-A)t}{\hbar}}\alpha(t)+(E_0+A)b(t)
	\end{align*}

	$I.$ lässt sich umformen wie folgt:
	\begin{align*}
		&&	i\hbar e^{-i\frac{(E_0-A)t}{\hbar}}\dot{\alpha}(t)&=-\hbar\omega_1 \cos(\omega t) e^{-i\frac{(E_0+A)t}{\hbar}}\beta(t) \\
		\Longleftrightarrow && 
			i\dot{\alpha}(t)&=-\omega_1\frac{e^{i\omega t}+e^{-i\omega t}}{2} e^{-i\frac{(E_0+A-(E_0-A))t}{\hbar}}\beta(t) \\
		\Longleftrightarrow && 
			2i\dot{\alpha}(t)&=-\omega_1(e^{i\omega t}+e^{-i\omega t})e^{-i\omega_0 t}\beta(t) \\
		\Longleftrightarrow &&
			2i\dot{\alpha}(t)&=-\omega_1\beta(t)(e^{i(\omega-\omega_0) t}+e^{-i(\omega+\omega_0) t})
	\end{align*}

	$II.$ lässt sich genauso umformen:
	\begin{align*}
		&&	i\hbar e^{-i\frac{(E_0+A)t}{\hbar}}\dot{\beta}(t)&=-\hbar\omega_1\cos(\omega t)e^{-i\frac{(E_0-A)t}{\hbar}} \alpha(t) \\
		\Longleftrightarrow && i\dot{\beta}(t)&=-\omega_1\alpha(t)\frac{e^{i\omega t}+e^{-i\omega t}}{2}e^{-i\frac{E_0-A-(E_0+A)t}{\hbar}} \\
		\Longleftrightarrow &&
			2i\dot{\beta}(t)&=-\omega_1\alpha(t)(e^{i\omega t}+e^{-i\omega t})e^{i\omega_0 t} \\
		\Longleftrightarrow &&
			2i\dot{\beta}(t)&=-\omega_1\alpha(t)(e^{i(\omega+\omega_0) t}+e^{-i(\omega-\omega_0) t})
	\end{align*}



\end{document}

% vim: spell spelllang=de tw=79
