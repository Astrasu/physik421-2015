\documentclass[11pt, ngerman, fleqn, DIV=15, headinclude]{scrartcl}

\usepackage[bibatend, color]{../header}

\hypersetup{
    pdftitle=
}

\renewcommand{\thesubsection}{\thesection.\alph{subsection}}

\usepackage{units}
\usepackage{listings}
\usepackage{beramono}
\lstset{
    basicstyle=\small\tt
}

%\subject{}
\title{Quantenmechanik, Blatt 6}
%\subtitle{}
\author{
    Frederike Schrödel \and Heike Herr \and Jan Weber \and Simon Schlepphorst
}



\begin{document}

\maketitle

\section{Operatoren - Formalismus}

\subsection{}

\begin{align*}
	\hat{\sigma}_1&=\begin{pmatrix}
					0 & 1\\
					1 & 0
					\end{pmatrix} &
	\hat{\sigma}_2&=\begin{pmatrix}
					0& -i \\
					i& 0
					\end{pmatrix} &
	\hat{\sigma}_3&=\begin{pmatrix}
					1 & 0 \\
					0 & -1
					\end{pmatrix}	
\end{align*}
\begin{itemize}
	\item
	Eigenvektoren von $\hat{\sigma}_1$ sind:
	$ \begin{pmatrix}1 \\ 1 \end{pmatrix} $ zu Eigenwert 1 und $ \begin{pmatrix} 1 \\ -1 \end{pmatrix} $ zu Eigenwert -1.
	
	Es gilt: 
	\begin{equation*}
		|\psi\rangle=\begin{pmatrix}1 \\0\end{pmatrix}=\frac{1}{2}\cdot\begin{pmatrix}1\\1\end{pmatrix}+\frac{1}{2}\cdot\begin{pmatrix}1\\-1\end{pmatrix}
	\end{equation*}
		
	\item
	Eigenvektoren von $\hat{\sigma}_2$ sind:
	$\begin{pmatrix}1\\-i\end{pmatrix}$ zu Eigenwert -1 und $\begin{pmatrix}1\\i\end{pmatrix}$ zu Eigenwert 1.

	Es gilt:
	\begin{equation*}
		|\psi\rangle=\frac{1}{2}\cdot\begin{pmatrix}1\\-i\end{pmatrix}+\frac{1}{2}\cdot\begin{pmatrix}1\\i\end{pmatrix}
	\end{equation*}

	\item
	Eigenvektoren von $\hat{\sigma}_3$ sind:
	$\begin{pmatrix}1\\0\end{pmatrix}$ zu Eigenwert 1 und $\begin{pmatrix}0\\1\end{pmatrix}$ zu Eigenwert -1.

	Es gilt:
	\begin{equation*}
		|\psi\rangle=
		1\cdot\begin{pmatrix}1\\0\end{pmatrix}+0\cdot\begin{pmatrix}0\\1\end{pmatrix}
	\end{equation*}

	\item Mittelwert von $|\psi\rangle$:
	\begin{align*}
		\langle\psi|\hat{\sigma}_1|\psi\rangle=
			&\begin{pmatrix}1&0\end{pmatrix}	 \begin{pmatrix}0&1\\1&0\end{pmatrix} \begin{pmatrix}1\\0\end{pmatrix}
			=\begin{pmatrix}1&0\end{pmatrix}\begin{pmatrix}0\\1\end{pmatrix}=0\\
		\langle\psi|\hat{\sigma}_2|\psi\rangle=&
			\begin{pmatrix}1&0\end{pmatrix}	 \begin{pmatrix}0&-i\\i&0\end{pmatrix} \begin{pmatrix}1\\0\end{pmatrix}=\begin{pmatrix}1&0\end{pmatrix}\begin{pmatrix}0\\i\end{pmatrix}=0\\
		\langle\psi|\hat{\sigma}_3|\psi\rangle=&
			\begin{pmatrix}1&0\end{pmatrix}	 \begin{pmatrix}1&0\\0&-1\end{pmatrix} \begin{pmatrix}1\\0\end{pmatrix}=\begin{pmatrix}1&0\end{pmatrix}\begin{pmatrix}1\\0\end{pmatrix}=1
	\end{align*}

\end{itemize}

\subsection{}

Ja, die Matrizen $\hat{A}$ und $\hat{B}$ sind hermitesch:
\begin{align*}
	\hat{A}=&\begin{pmatrix}1&0&0\\0&1&0\\0&0&-1\end{pmatrix}=\hat{A}^{\dag} & \hat{B}=&\begin{pmatrix}0&i&0\\-i&0&0\\0&0&-1\end{pmatrix}=\hat{B}^{\dag}
\end{align*}
Die Eigenwerte von $\hat{A}$ sind 1 und -1 und möglichen Eigenvektorbasis $\begin{pmatrix}\begin{pmatrix}1\\0\\0\end{pmatrix}, \begin{pmatrix}0\\1\\0\end{pmatrix} , \begin{pmatrix}0\\0\\1\end{pmatrix}\end{pmatrix}$, wobei die ersten beiden Vektoren den Eigenraum zum Eigenwert 1 aufspannen und der dritte den zum Eigenwert -1 (ablesbar, da Diagonalmatrix).

Die Eigenwerte von $\hat{B}$ sind auch 1 und -1 und eine mögliche Eigenvektorbasis ist $\begin{pmatrix} \begin{pmatrix}1\\-i\\0\end{pmatrix},\begin{pmatrix}1\\i\\0\end{pmatrix},\begin{pmatrix}0\\0\\1\end{pmatrix}\end{pmatrix}$.
	Diese kann man ablesen, da $\hat{B}$ sich aufteilt in $-\hat{\sigma}_2$ und -1 auf der Diagonalen. Daher entsprechen die Eigenwerte den negativen von $\hat{\sigma}_2$ und -1.

Die Eigenvektoren zum Eigenwert 1 werden aufgespannt durch den Eigenvektoren von $\hat{\sigma}$ zum Eigenwert -1 mit einer zusätzlichen $0$ als dritten Eintrag und die Eigenvektoren von $\hat{B}$ zum Eigenwert -1 werden aufgespannt von dem Eigenvektor von $\hat{\sigma}_2$ zum Eigenwert 1 mit zusätzlicher $0$ als dritten Eintrag und dem dritten Eigenvektor $\begin{pmatrix}0\\0\\1\end{pmatrix}$, der dem letzten Diagonaleintrag entspricht. 

Da die beiden Eigenvektoren von $\hat{\sigma}_2$ orthogonal zueinander sind und der dritte Vektor in der Basis offensichtlich orthogonal zu den anderen beiden ist, ist die Basis eine Orthogonalbasis. Wir erhalten eine Orthonormalbasis durch Normierung:
\[
\begin{pmatrix}\frac{1}{\sqrt{2}}\begin{pmatrix}1\\-i\\0\end{pmatrix}, \frac{1}{\sqrt{2}}\begin{pmatrix}1\\i\\0\end{pmatrix},	\begin{pmatrix}0\\0\\1\end{pmatrix}\end{pmatrix}
\]

Somit erhalten wir die Spektralzerlegung von $\hat{B}$:
\begin{align*}
	\hat{B}=&1\cdot \frac{1}{2}\begin{pmatrix}1\\-i\\0\end{pmatrix} \begin{pmatrix}1&i&0\end{pmatrix}+(-1)\cdot\frac{1}{2}\begin{pmatrix}1\\i\\0\end{pmatrix}\begin{pmatrix}1&-i&0\end{pmatrix}+ (-1)\cdot \begin{pmatrix} 0\\0\\1\end{pmatrix}\begin{pmatrix}0&0&1\end{pmatrix} \\
		=&1\cdot\frac{1}{2}\begin{pmatrix}1&i&0\\-i&1&0\\0&0&0\end{pmatrix} +(-1)\cdot\frac{1}{2}\begin{pmatrix} 1 &- i & 0\\ i & 1 & 0 \\ 0&0&0\end{pmatrix}+ (-1)\cdot\begin{pmatrix}0&0&0\\0&0&0\\0&0&1\end{pmatrix}
\end{align*}

\subsection{}

Wir wissen aus der linearen Algebra, dass $\hat{B}$ Diagonalform hat nach konjugieren mit einer Basistransformation zu einer Basis von Eigenvektoren zu $\hat{B}$. Weiter wissen wir auch, dass die Basistransformation zwischen Orthonormalbasen eine unitäre Transformation ist, als auch dass von unitären invertierbaren Matrizen das Inverse die konjugierte transponierte Matrix ist. Somit erfüllt
\[
\hat{U}=\begin{pmatrix} \frac{1}{\sqrt{2}} & \frac{1}{\sqrt{2}} & 0 \\ -\frac{1}{\sqrt{2}}i & \frac{1}{\sqrt{2}}i & 0 \\ 0 & 0 &1\end{pmatrix}
\]
das gewünschte (die Spalten von $\hat{U}$ sind die orthonormierte Eigenvektorbasis von $\hat{B}$).

\subsection{}

Ja, $\hat{A}$ und $\hat{B}$ kommutieren, da  $\hat{A}$ sich aufteilt in die Einheitsmatrix $\hat{I}_2$ und -1 auf der Diagonalen. Die Einheitsmatrix kommutiert mit $-\hat{\sigma}_2$, da es eine Einheitsmatrix ist, und -1 kommutiert mit -1, da die Matrizen gleich sind (als auch beide diagonal).


\section{Projektoren}

Gegeben ist eine Hilbert-Basis durch $\ket{n}$, $\del{n=1,...,N}$ im Hilbertraum $H$. Der zugehörige Projektor ist gegeben durch $\hat{P}_n=\ket{n}\bra{n}$. Es sollen mehrere Eigenschaften von $\hat{P}_n$ gezeigt werden.

\subsection{}

Es soll gezeigt werden, dass $\hat{P}_n^2=\hat{P}_n$ ist:
\begin{align*}
	\hat{P}_n^2	&= \del{\ket{n}\bra{n}}^2\\
				&= \ket{n}\underbrace{\left<n|n\right>}_{=1}\bra{n}\\
				&= \ket{n}\bra{n}\\
				&= \hat{P}_n
\end{align*}

\subsection{}

Es soll gezeigt werden, dass $\sum_{n=1}^{N}\hat{P}_n=\hat{I}$, wobei $\hat{I}$ die Identität von $H$ ist. Sei $\ket{x}$ ein Zustand in $H$. Dann kann dieser in der Basis $\ket{n}$ geschrieben werden als $\sum_{n=1}^{N}c_n\ket{n}$. Wendet man die Summe aller Projektoren auf diesen Zustand an, so folgt:
\begin{align*}
	\sum_{n=1}^{N}\hat{P}_n\ket{x}	&= \sum_{n=1}^{N}\ket{n}\bra{n}\sum_{m=1}^{N}c_m\ket{m}\\
									&= \sum_{n,m=1}^{N}c_m\ket{n}\underbrace{\left<n|m\right>}_{=\delta_{nm}}\\
									&= \sum_{n=1}^{N}c_n\ket{n}\\
									&=	\ket{x}
\end{align*}

\subsection{}

Gegeben sind die beiden Matrizen
\begin{align*}
	\begin{pmatrix}
		1&0\\
		0&0
	\end{pmatrix}
\end{align*}
und
\begin{align*}
	\begin{pmatrix}
		0&0\\
		0&1
	\end{pmatrix}
\end{align*}
Es soll gezeigt werden, dass diese Projektoren sind, also als $\ket{n}\bra{n}$ geschrieben werden können.\\
Offensichtlich gilt:
\begin{align*}
	\begin{pmatrix}
		1&0\\
		0&0
	\end{pmatrix}
	=
	\begin{pmatrix}
		1\\
		0
	\end{pmatrix}
	\begin{pmatrix}
		1&0
	\end{pmatrix}
\end{align*}
und
\begin{align*}
	\begin{pmatrix}
		0&0\\
		0&1
	\end{pmatrix}
	=
	\begin{pmatrix}
		0\\
		1
	\end{pmatrix}
	\begin{pmatrix}
		0&1
	\end{pmatrix}
\end{align*}
Die beiden Matrizen projizieren also auf die Zustände $\begin{pmatrix}1\\0\end{pmatrix}$ und $\begin{pmatrix}0\\1\end{pmatrix}$, welche eine Basis eines 2-Niveau-Systems bilden.\\
Ihre Unterräume sind $\cbr{k\begin{pmatrix}1\\0\end{pmatrix}|k\in\C}$ und $\cbr{k'\begin{pmatrix}0\\1\end{pmatrix}|k'\in\C}$.

\subsection{}

Gegeben ist die Matrix:
\begin{align*}
	\begin{pmatrix}
		\nicefrac12&\nicefrac12\\
		\nicefrac12&\nicefrac12
	\end{pmatrix}
\end{align*}
Es soll überprüft werden ob diese ein Projektor ist.\\
Man sieht:
\begin{align*}
	\begin{pmatrix}
		\nicefrac12&\nicefrac12\\
		\nicefrac12&\nicefrac12
	\end{pmatrix}
	=
	\begin{pmatrix}
		\nicefrac1{\sqrt{2}}\\
		\nicefrac1{\sqrt{2}}
	\end{pmatrix}
	\begin{pmatrix}
		\nicefrac1{\sqrt{2}}&\nicefrac1{\sqrt{2}}
	\end{pmatrix}
\end{align*}
Die Matrix projiziert also auf den Zustand $\begin{pmatrix}\nicefrac1{\sqrt{2}}\\\nicefrac1{\sqrt{2}}\end{pmatrix}$.\\
Der zugehörige Unterraum ist $\cbr{k\begin{pmatrix}\nicefrac1{\sqrt{2}}\\\nicefrac1{\sqrt{2}}\end{pmatrix}|k\in\C}$.


\section{Matrixdarstellung von Operatoren}

\subsection{ }
Die gesuchte Matrix ist:
\begin{align*}
	\hat A = \begin{pmatrix}
		5 & \alpha & 0\\
		\beta & 0 & \iup\\
		0 & -\iup & \gamma
	\end{pmatrix}
\end{align*}

\subsection{ }
Damit $\hat A$ hermitesch ist muss gelten:
\begin{align*}
	\hat A = \hat A^\dagger
\end{align*}
Das ist gegeben für $\gamma \in \mathbb R$ und $\alpha, \beta \in \mathbb C$
mit $\alpha^* = \beta$.

\subsection{ }
Im folgenden wird $\alpha = \beta = 0$ gesetzt und die Matrix $\hat A$ genutzt.
\begin{align*}
	\hat A = \begin{pmatrix}
		5 & 0 & 0\\
		0 & 0 & \iup\\
		0 & -\iup & \gamma
	\end{pmatrix}
\end{align*}

Es sind die Eigenvektoren zu bestimmen.
\begin{align*}
	\det\del{\hat A - \lambda \mathbb 1} &=
	\begin{vmatrix}
		5-\lambda & 0 & 0\\
		0 & -\lambda & \iup\\
		0 & -\iup & \gamma-\lambda
	\end{vmatrix}\\
	&= \del{\lambda - 5} \del{\lambda - \del{\frac\gamma2 + \frac12
		\sqrt{\gamma^2 + 4}}} \del{\lambda - \del{\frac\gamma2 - \frac12
			\sqrt{\gamma^2 + 4}}}
			\overset{!}{=} 0
	\intertext{Setze $\gamma_{\pm}\equiv-\frac\iup2\gamma \pm \frac\iup2 \sqrt{\gamma^2 + 4}$. Damit ergeben sich die normierten Eigenvektoren}
	\vec v_1 &= \begin{pmatrix}
		1\\
		0\\
		0
	\end{pmatrix}\\
	\vec v_2 &= \frac1{\sqrt{1+\abs{\gamma_+}^2}}
	\begin{pmatrix}
		0\\
		\gamma_+\\
		1
	\end{pmatrix}\\
	\vec v_3 &= \frac1{\sqrt{1+\abs{\gamma_-}^2}}
	\begin{pmatrix}
		0\\
		\gamma_-\\
		1
	\end{pmatrix}\\
\end{align*}

\subsection{ }
\begin{align*}
	\Bracket{\hat A} &= \Braket{\psi^* | \hat A | \psi} = \begin{pmatrix}
		a^* & b^* & 0
	\end{pmatrix} \begin{pmatrix}
		5 & 0 & 0\\
		0 & 0 & \iup\\
		0 & -\iup & \gamma
	\end{pmatrix} \begin{pmatrix}
		a \\
		b \\
		0
	\end{pmatrix} \\
	&= 5\abs{a}^2
\end{align*}

\subsection{}

Das System befinde sich im Zustand $\ket{\psi}$. Misst man den Operator $\hat{A}$, so können als Ergebnis nur die Eigenwerte von $\hat{A}$ auftreten. Die Wahrscheinlichkeit diese zu messen ist gegeben durch
\begin{align*}
	P\del{v_1}	&= \abs{
		\begin{pmatrix}
			a\\b\\0
		\end{pmatrix}
		\begin{pmatrix}
			1\\0\\0
		\end{pmatrix}
	}^2\\
				&= \abs{a}^2\\
	P\del{v_2}	&= \frac1{1+\abs{\gamma_+}^2}\abs{
		\begin{pmatrix}
			a\\b\\0
		\end{pmatrix}
		\begin{pmatrix}
			0\\\gamma_+\\1
		\end{pmatrix}
	}^2\\
				&= \frac{\abs{b\gamma_+}^2}{1+\abs{\gamma_+}^2}\\
	P\del{v_3}	&= \frac1{1+\abs{\gamma_-}^2}\abs{
		\begin{pmatrix}
			a\\b\\0
		\end{pmatrix}
		\begin{pmatrix}
			0\\\gamma_-\\1
		\end{pmatrix}
	}^2\\
				&= \frac{\abs{b\gamma_-}^2}{1+\abs{\gamma_-}^2}\\
\end{align*}

\subsection{ }

Da $\hat{A}$ auf alle drei Zustände abbildet kann das System entweder im Zustand $\ket{1}$ oder einer Überlagerung von $\ket{2}$ und $\ket{3}$ sein, je nachdem welcher Eigenwert von $\hat{A}$ gemessen worden ist.


\section{Zwei-Niveau System}

\subsection{}

Betrachtet wird $\hat{H}=A\hat{\sigma}_3$, $A\in\R$. Es soll der Zeitentwicklungsoperator als Linearkombination von $\hat{I}$ und $\hat{\sigma}_i$ dargestellt werden.\\
Es gilt:
\begin{align*}
	\hat{\sigma}_i^2=
	\begin{pmatrix}
		1&0\\
		0&1
	\end{pmatrix}
	=\hat{I}
\end{align*}
Der Zeitentwicklungsoperator ist gegeben durch:
\begin{align*}
	u\del{t}=\eup^{-\iup\frac{A}{\hbar}\hat{\sigma}_3t}
\end{align*}
Setzte $x\equiv\frac{A}{\hbar}t$. Die Exponentialfunktion ist durch die Reihenentwicklung gegeben:
\begin{align*}
	\eup^{-\iup\frac{A}{\hbar}\hat{\sigma}_3t}	&= \sum_{n=0}^{\infty}\frac{\del{-\iup x\hat{\sigma}_3}^n}{n!}\\
												&= \hat{I} - \iup x\hat{\sigma}_3 - \frac{x^2}{2}\hat{I} + \iup\frac{x^3}{6}\hat{\sigma}_3 + \frac{x^4}{24}\hat{I} - \iup\frac{x^5}{120}\hat{\sigma}_3 + \mathbb{O}\del{x^6}\\
												&= \del{1 - \frac{x^2}{2} + \frac{x^4}{24} + \mathbb{O}\del{x^6}}\hat{I} - \iup\del{x - \frac{x^3}{6} + \frac{x^5}{120} + \mathbb{O}\del{x^7}}\hat{\sigma}_3\\
												&= \del{\sum_{n=0}^{\infty}\del{-1}^n\frac{x^{2n}}{\del{2n}!}}\hat{I} - \iup\del{\sum_{n=0}^{\infty}\del{-1}^n\frac{x^{2n+1}}{\del{2n+1}!}}\hat{\sigma}_3\\
												&= \cos\del{x}\hat{I} - \iup\sin\del{x}\hat{\sigma}_3\\
												&= \cos\del{\frac{A}{\hbar}t}\hat{I} - \iup\sin\del{\frac{A}{\hbar}t}\hat{\sigma}_3\\
\end{align*}






























\end{document}

% vim: spell spelllang=de tw=79
