\documentclass[11pt, ngerman, fleqn, DIV=15, headinclude]{scrartcl}

\usepackage[bibatend, color]{../header}

\hypersetup{
    pdftitle=
}

\renewcommand{\thesubsection}{\thesection.\alph{subsection}}

\usepackage{units}
\usepackage{listings}
\usepackage{beramono}
\lstset{
    basicstyle=\small\tt
}

%\subject{}
\title{Quantenmechanik, Blatt 8}
%\subtitle{}
\author{
    Frederike Schrödel \and Heike Herr \and Jan Weber \and Simon Schlepphorst
}



\begin{document}

\maketitle
\begin{center}
	\begin{tabular}{l|c|c|c|c|c}
		Aufgabe &1&2&3&4&$\Sigma$\\
		\hline
		Punkte &\quad /12 & \quad /20 & \quad /13 & \quad /7 & \quad /52
	\end{tabular}\\
\end{center}

\section{Kommutatoren}

\section{Harmonischer Oszillator in zwei Dimensionen}

\section{Virialtheorem}

\section{Angeregte Zustände des Harmonischen Oszillators}


\end{document}

% vim: spell spelllang=de tw=79
