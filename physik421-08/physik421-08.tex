\documentclass[11pt, ngerman, fleqn, DIV=15, headinclude]{scrartcl}

\usepackage[bibatend, color]{../header}

\hypersetup{
    pdftitle=
}

\renewcommand{\thesubsection}{\thesection.\alph{subsection}}

\usepackage{units}
\usepackage{listings}
\usepackage{beramono}
\lstset{
    basicstyle=\small\tt
}

%\subject{}
\title{Quantenmechanik, Blatt 8}
%\subtitle{}
\author{
    Frederike Schrödel \and Heike Herr \and Jan Weber \and Simon Schlepphorst
}



\begin{document}

\maketitle
\begin{center}
	\begin{tabular}{l|c|c|c|c|c}
		Aufgabe &1&2&3&4&$\Sigma$\\
		\hline
		Punkte &\quad /12 & \quad /20 & \quad /13 & \quad /7 & \quad /52
	\end{tabular}\\
\end{center}

\section{Kommutatoren}

\subsection{ }

	$\sbr{H,A}=0$ und $\sbr{H,B}=0$ impliziert, dass Basen $\mathcal{A}$ und $\mathcal{B}$ existieren, wobei $\mathcal{A}$ aus gemeinsamen Eigenvektoren von $H$ und $A$ besteht, $\mathcal{B}$ aus gemeinsamen Eigenvektoren von $H$ und $B$.

	Angenommen die Eigenwerte von $H$ sind nicht entartet, d.h. die Eigenräume von $H$ eindimensional sind. 

	Die Basen $\mathcal{A}$ und $\mathcal{B}$ bestehen somit aus Eigenvektoren von $H$, die sich innerhalb der Eigenräume nur bis auf Skalare Vorfaktoren unterscheiden. Somit sind die Vektoren in $\mathcal{A}$ auch Eigenvektoren von $B$. 

	Somit ist $\mathcal{A}$ eine gemeinsame Basis aus Eigenvektoren von $A$ und $B$. Das impliziert allerdings, dass $A$ und $B$ kommutieren, also $\sbr{A,B}=0$. Das ist ein Widerspruch zur Voraussetzung.

	Somit existiert mindetens ein entarteter Eigenwert von $H$.

\subsection{ }
\begin{align*}
	\sbr{A,B^n} &= B\sbr{A,B^{n-1}} + \sbr{A,B}B^{n-1}\\
	&= B\del{B\sbr{A,B^{n-2}} + \sbr{A,B}B^{n-2}} + 1\sbr{A,B}B^{n-1}\\
	&= B^2 \sbr{A,B^{n-2}} + B^1 \sbr{A,B}B^{n-2} + B^0 \sbr{A,B}B^{n-1}\\
	&= \sum_{s=0}^{n-1} B^s \sbr{A,B} B^{n-s-1}
\end{align*}

\subsection{ }
Zu zeigen ist die Baker-Cambell-Hausdorff Formel $\eup^A\eup^B =
\eup^{A+B}\eup^{\sbr{A,B}/2}$, wenn $\sbr{\sbr{A,B},A} = \sbr{\sbr{A,B},B} = 0$
gilt.

Sei $F\del t = \eup^{tA}\eup^{tB}$:
\begin{align*}
	\od{F}{t} &= A\eup^{tA}\eup^{tB} + \eup^{tA}B\eup^{tB}\\
	&= A\eup^{tA}\eup^{tB} + \sum_{n=0}^\infty \frac{t^n}{n!}
	A^n B \eup^{tB}\\
	&= A\eup^{tA}\eup^{tB} + \sum_{n=0}^\infty \frac{t^n}{n!}\del{BA^n +
	\sbr{A^n,B}} \eup^{tB}\\
	&= A\eup^{tA}\eup^{tB} + B\eup^{tA}\eup^{tB} + \sum_{n=0}^\infty
	\frac{t^n}{n!} n \sbr{A,B} A^{n-1}\eup^{tB}\\
	&= A\eup^{tA}\eup^{tB} + B\eup^{tA}\eup^{tB} + t\sbr{A,B}\sum_{n=0}^\infty
	\frac{t^{n-1}}{\del{n-1}!} A^{n-1}\eup^{tB}\\
	&= A\eup^{tA}\eup^{tB} + B\eup^{tA}\eup^{tB} +
	t\sbr{A,B}\eup^{tA}\eup^{tB}\\
	&= \del{A + B + t\sbr{A,B}}\eup^{tA}\eup^{tB}
\end{align*}
\begin{align*}
	\implies &\int_0^1 \dif F = \int_0^1 \del{A + B +
		t\sbr{A,B}}\eup^{tA}\eup^{tB} \dif t\\
		\iff &\sbr{\eup^{tA}\eup^{tB}}_0^1 =
		\sbr{\eup^{tA+tB+\frac{t^2}2 \sbr{A,B}}}_0^1\\
		\iff &\eup^A\eup^B = \eup^{A+B}\eup^{\frac12\sbr{A,B}}
\end{align*}

\section{Harmonischer Oszillator in zwei Dimensionen}

\section{Virialtheorem}

\section{Angeregte Zustände des Harmonischen Oszillators}

	Wir haben gegeben:
	\[ \psi(X)=(2X^3-3X)e^{-\frac{X^2}{2}} \]
	\[ \hat{H}=\hat{a}^{\dagger}\hat{a}+\frac{1}{2}=\frac{1}{2}+\frac{1}{2} (\hat{X}-\od{}{X})(\hat{X}+\od{}{X}) \]

	Betrachte nun $\hat{H}\psi$:
	\begin{align*}
	\hat{H}\psi(X)&= \frac{1}{2} \psi(X) + \frac{1}{2} (\hat{X}-\od{}{X})(\hat{X}+\od{}{X}) \psi(X) \\
		&=\frac{1}{2}\psi(X) + \frac{1}{2}(\hat{X}-\od{}{X})(2X^4-3X^2+6X^2-3-2X^4+3X^2)e^{-\frac{X^2}{2}} \\
		&=\frac{1}{2}\psi(X) + \frac{3}{2}(\hat{X}-\od{}{X})(2X^2-1)e^{-\frac{X^2}{2}} \\
		&=\frac{1}{2}\psi(X) + \frac{3}{2}(2X^3-X-4X+2X^3-X) e^{-\frac{X^2}{2}} \\
		&=\frac{1}{2}\psi(X) + 3 (2X^3-3X) e^{-\frac{X^2}{2}} \\
		&= \frac{7}{2} \psi(X)
	\end{align*}

	Das heißt, $\psi$ ist ein Eigenvektor des Hamiltonoperators zum Eigenwert $\frac{7}{2}$. Da die Eigenwerte des Hamiltonoperators nicht entartet sind, ist $\psi$ ein Vielfaches des normierten Eigenzustands $|3\rangle$, also $\psi=\lambda |3\rangle$ für ein $\lambda\in\mathbb{C}$ und $|\lambda|=|\psi|$. 
	$\psi$ hat dann die Energie $|\psi|\cdot E_3=|\psi| \hbar \omega \frac{7}{2}$.

\end{document}

% vim: spell spelllang=de tw=79
