\documentclass[11pt, ngerman, fleqn, DIV=15, headinclude]{scrartcl}

\usepackage[bibatend, color]{../header}

\hypersetup{
    pdftitle=
}

\renewcommand{\thesubsection}{\thesection.\alph{subsection}}

\usepackage{units}
\usepackage{listings}
\usepackage{beramono}
\lstset{
    basicstyle=\small\tt
}

%\subject{}
\title{Quantenmechanik, Blatt 9}
%\subtitle{}
\author{
    Frederike Schrödel \and Heike Herr \and Jan Weber \and Simon Schlepphorst
}



\begin{document}

\maketitle
\begin{center}
	\begin{tabular}{l|c|c|c|c|c|c}
		Aufgabe &1&2&3&4&5&$\Sigma$\\
		\hline
		Punkte &\quad /9 & \quad /15 & \quad /9 & \quad /14 & \quad
		/11& \quad /58
	\end{tabular}\\
\end{center}


\section{Der zweidimensionale harmonische Oszillator}

\section{Ein Elektron in der Falle}

\section{Ein geladenes Teilchen im Magnetfeld}

\section{Der Schwerpunkt und die Relativbewegung}

\section{Drehimpuls}



\end{document}

% vim: spell spelllang=de tw=79
